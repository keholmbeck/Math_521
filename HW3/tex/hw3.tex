%%%%%%%%%%%%%%%%%%%%%%%%%%%%%%%%%%%%%%%%%
% Programming/Coding Assignment
% LaTeX Template
%
% This template has been downloaded from:
% http://www.latextemplates.com
%
% Original author:
% Ted Pavlic (http://www.tedpavlic.com)
%
% Note:
% The \lipsum[#] commands throughout this template generate dummy text
% to fill the template out. These commands should all be removed when 
% writing assignment content.
%
% This template uses a Perl script as an example snippet of code, most other
% languages are also usable. Configure them in the "CODE INCLUSION 
% CONFIGURATION" section.
%
%%%%%%%%%%%%%%%%%%%%%%%%%%%%%%%%%%%%%%%%%

%----------------------------------------------------------------------------------------
%	PACKAGES AND OTHER DOCUMENT CONFIGURATIONS
%----------------------------------------------------------------------------------------

\documentclass{article}

\usepackage{fancyhdr} % Required for custom headers
\usepackage{lastpage} % Required to determine the last page for the footer
\usepackage{extramarks} % Required for headers and footers
\usepackage[usenames,dvipsnames]{color} % Required for custom colors
\usepackage{graphicx} % Required to insert images
\usepackage{listings} % Required for insertion of code
\usepackage{courier} % Required for the courier font
\usepackage{lipsum} % Used for inserting dummy 'Lorem ipsum' text into the template

\usepackage{float}
\usepackage{amsfonts}
\usepackage{amsmath}

% Margins
\topmargin=-0.45in
\evensidemargin=0in
\oddsidemargin=0in
\textwidth=6.5in
\textheight=9.0in
\headsep=0.25in

\linespread{1.1} % Line spacing

% Set up the header and footer
\pagestyle{fancy}
\lhead{\hmwkAuthorName} % Top left header
\chead{\hmwkClass\ : \hmwkTitle} % Top center head
\rhead{\firstxmark} % Top right header
\lfoot{\lastxmark} % Bottom left footer
\cfoot{} % Bottom center footer
\rfoot{Page\ \thepage\ of\ \pageref{LastPage}} % Bottom right footer
\renewcommand\headrulewidth{0.4pt} % Size of the header rule
\renewcommand\footrulewidth{0.4pt} % Size of the footer rule

\setlength\parindent{0pt} % Removes all indentation from paragraphs

%----------------------------------------------------------------------------------------
%	CODE INCLUSION CONFIGURATION
%----------------------------------------------------------------------------------------

\usepackage{color} %red, green, blue, yellow, cyan, magenta, black, white
\definecolor{mygreen}{RGB}{28,172,0} % color values Red, Green, Blue
\definecolor{mylilas}{RGB}{170,55,241}

\lstset{language=Matlab,%
    basicstyle=\ttfamily\footnotesize,breaklines=true
    %basicstyle=\footnotesize\color{red},
    breaklines=true,%
    xleftmargin=0.5in,
    %xrightmargin=0.25in,
    morekeywords={matlab2tikz},
    keywordstyle=\color{blue},%
    morekeywords=[2]{1}, keywordstyle=[2]{\color{black}},
    identifierstyle=\color{black},%
    stringstyle=\color{mylilas},
    commentstyle=\color{mygreen},%
    showstringspaces=false,%without this there will be a symbol in the places where there is a space
    numbers=left,%
    numberstyle={\tiny \color{black}},% size of the numbers
    numbersep=9pt, % this defines how far the numbers are from the text
    emph=[1]{for,end,break},emphstyle=[1]\color{blue}, %some words to emphasise
    %emph=[2]{word1,word2}, emphstyle=[2]{style},    
}


%----------------------------------------------------------------------------------------
%	DOCUMENT STRUCTURE COMMANDS
%	Skip this unless you know what you're doing
%----------------------------------------------------------------------------------------

% Header and footer for when a page split occurs within a problem environment
\newcommand{\enterProblemHeader}[1]{
%\nobreak\extramarks{#1}{#1 continued on next page\ldots}\nobreak
%\nobreak\extramarks{#1 (continued)}{#1 continued on next page\ldots}\nobreak
}

% Header and footer for when a page split occurs between problem environments
\newcommand{\exitProblemHeader}[1]{
\nobreak\extramarks{#1 (continued)}{#1 continued on next page\ldots}\nobreak
\nobreak\extramarks{#1}{}\nobreak
}

\setcounter{secnumdepth}{0} % Removes default section numbers
\newcounter{homeworkProblemCounter} % Creates a counter to keep track of the number of problems

\newcommand{\homeworkProblemName}{}
\newenvironment{homeworkProblem}[1][Problem \arabic{homeworkProblemCounter}]{ % Makes a new environment called homeworkProblem which takes 1 argument (custom name) but the default is "Problem #"
\stepcounter{homeworkProblemCounter} % Increase counter for number of problems
\renewcommand{\homeworkProblemName}{#1} % Assign \homeworkProblemName the name of the problem
\subsection{\homeworkProblemName} % Make a section in the document with the custom problem count
\enterProblemHeader{\homeworkProblemName} % Header and footer within the environment
}{
\exitProblemHeader{\homeworkProblemName} % Header and footer after the environment
}

\newcommand{\problemAnswer}[1]{ % Defines the problem answer command with the content as the only argument
\noindent\framebox[\columnwidth][c]{\begin{minipage}{0.98\columnwidth}#1\end{minipage}} % Makes the box around the problem answer and puts the content inside
}

\newcommand{\homeworkSectionName}{}
\newenvironment{homeworkSection}[1]{ % New environment for sections within homework problems, takes 1 argument - the name of the section
\renewcommand{\homeworkSectionName}{#1} % Assign \homeworkSectionName to the name of the section from the environment argument
\subsection{\homeworkSectionName} % Make a subsection with the custom name of the subsection
\enterProblemHeader{\homeworkProblemName\ [\homeworkSectionName]} % Header and footer within the environment
}{
\enterProblemHeader{\homeworkProblemName} % Header and footer after the environment
}


%----------------------------------------------------------------------------------------
%   NAME AND CLASS SECTION
%----------------------------------------------------------------------------------------

\newcommand{\hmwkTitle}{Homework\ 3} % Assignment title
\newcommand{\hmwkDueDate}{Thursday,\ March\ 22,\ 2018} % Due date
\newcommand{\hmwkClass}{Math\ 521} % Course/clas
\newcommand{\hmwkAuthorName}{Kristin Holmbeck} % Your name

%----------------------------------------------------------------------------------------
%   TITLE PAGE
%----------------------------------------------------------------------------------------

\title{
\textmd{\textbf{\hmwkClass \ \hmwkTitle}}\\
\normalsize\vspace{0.1in}\small{Due\ on\ \hmwkDueDate}\\
\vspace{0.1in}
\vspace{0.2in}
}

\author{\textbf{\hmwkAuthorName}}
\date{} % Insert date here if you want it to appear below your name

%----------------------------------------------------------------------------------------

\begin{document}

\maketitle

%----------------------------------------------------------------------------------------
%   TABLE OF CONTENTS
%----------------------------------------------------------------------------------------

%\setcounter{tocdepth}{1} % Uncomment this line if you don't want subsections listed in the ToC
\vspace{0.75in}
\tableofcontents
\listoffigures
\newpage

%----------------------------------------------------------------------------------------
%   PROBLEM 1
%----------------------------------------------------------------------------------------

% To have just one problem per page, simply put a \clearpage after each problem

\begin{section}{Theory}

\begin{homeworkSection}{1. Gradient of Inner Product}

Show that $\nabla_{\bm{v}} (\bm{v}, \bm{v}) = 2\bm{v}$ and that for a symmetric matrix $C$, $\nabla_{\bm{v}} (\bm{v}, C\bm{v}) = 2C\bm{v}$. 

\problemAnswer{
Assume $\bm{v} \in \mathbb{R}^n$.
	\begin{align*}
		\nabla_{\bm{v}} (\bm{v}, \bm{v}) &= \nabla_{\bm{v}} \bm{v}^T\bm{v} \\
			&= \nabla_{\bm{v}} \left ( v_1^2 + v_2^2 + \cdots + v_n^2 \right ) \\
			&= \left( \frac{\partial \left ( v_1^2 + v_2^2 + \cdots + v_n^2 \right ) }{\partial v_1}, \ldots, \frac{\partial \left ( v_1^2 + v_2^2 + \cdots + v_n^2 \right ) }{\partial v_n}, \right ) \\ 
			&= \left( 2v_1,  2v_2, \ldots, 2v_n \right ) \\
			&= 2 \bm{v}
	\end{align*}
Next, show $\nabla_{\bm{v}} (\bm{v}, C\bm{v}) = 2C\bm{v}$ :
	\begin{align*}
		\text{Let } \bm{v} &= \begin{bmatrix} v_1 \\ v_2 \\ \vdots \\ v_n \end{bmatrix} \\
		\text{and } C &= \begin{bmatrix} c_{11} && c_{12} && \cdots && c_{1n} \\
			c_{21} && c_{22} && \cdots && c_{2n} \\ 
			\vdots && 		&& 			&& \vdots \\
			c_{n1} &&    \cdots && && c_{nn} \end{bmatrix}
			= \begin{bmatrix} c_{11} && c_{21} && \cdots && c_{n1} \\
			c_{12} && c_{22} && \cdots && c_{n2} \\ 
			\vdots && 		&& 			&& \vdots \\
			c_{1n} &&    \cdots && && c_{nn} \end{bmatrix} \\
		&= \begin{bmatrix} \vec{c}_1 \cdots \vec{c}_n \end{bmatrix} \\
		\text{then } \bm{v}^T C\bm{v} &= \begin{bmatrix} v_1 &&  \cdots && v_n \end{bmatrix}
				\begin{bmatrix} \vec{c}_1 \cdot \bm{v} \\ \vdots \\ \vec{c}_n \cdot \bm{v} \end{bmatrix}
			= v_1 \vec{c}_1 \cdot \bm{v} + \cdots + v_ n \vec{c}_n \cdot \bm{v} \\
		&= v_1 (c_{11} v_1 + c_{12} v_2 + \cdots + c_{1n}v_n)  + \cdots + v_n (c_{n1} v_1 + c_{n2} v_2 + \cdots + c_{nn}v_n) \\ 
		\text{So } \nabla_{\bm{v}} (\bm{v}, C\bm{v}) &= \nabla_{\bm{v}} \bm{v}^T C\bm{v} \\
		\nabla_{\bm{v}} \bm{v}^T C\bm{v} &= 
			\begin{bmatrix} \partial v_1 (\bm{v}^T C\bm{v}) \\ \vdots \\ \partial v_n (\bm{v}^T C\bm{v}) \end{bmatrix} \\
		&= \begin{bmatrix} \vec{c}_1 \cdot \bm{v} + \end{bmatrix} \\
		\partial v_k (\bm{v}^T C\bm{v}) &= v1 c_{1k} + v2 c_{2k} + \cdots + \partial v_k \left [ v_k (c_{k1}v1 + \cdots + c_{kk}v_k + \cdots + c_{kn}v_n) \right ] + \cdots + v_n c_{nk}
		\\
		\partial v_k (\bm{v}^T C\bm{v}) &= v1 c_{1k} + v2 c_{2k} + \cdots + \left [ (c_{k1}v1 + \cdots + c_{kk}v_k + \cdots + c_{kn}v_n ) + v_k c_{kk} \right ] + \cdots + v_n c_{nk} \\
		&= 2 \vec{c}_k \bm{v} \\
		\implies \nabla_{\bm{v}} (\bm{v}, C\bm{v}) &= 2 C \bm{v} \\
	\end{align*}
}

\end{homeworkSection}

\begin{homeworkSection}{2. Commutativity of Symmetric Matrix in the Inner Product}
Show that for a symmetric matrix $C$, $( \phi^{(1)}, C\phi^{(2)} ) = ( C\phi^{(1)}, \phi^{(2)} )$.
\\
\\
\problemAnswer{ 
	\begin{align*}
	(x,Cy) &= (Cy)^Tx = y^TC^T x = y^T Cx = (Cx,y) \\
	\implies (x,Cy) &= (Cx,y)
	\end{align*}
}

\end{homeworkSection}


\begin{homeworkSection}{3. Eigenvalues and eigenvectors}
$ X = \begin{bmatrix} -2 && -1 && 1 \\ 0 && -1 && 0 \\ -1 && 1 && 2 \\ 1 && -1 && 1
\end{bmatrix} $
\end{homeworkSection}


\begin{homeworkSection}{4. FLOP count}
Assume $A$ is $N \times P$, and that $N > P$. Then the SVD of $A$ is $A=U \Sigma V^T$ where $U$ is $N \times P$, $\Sigma$ is $P\times P$, and $V$ is $P\times P$.

\problemAnswer {
	\begin{align*}
	 A = \Sigma V^T &= 
		 \begin{bmatrix} \sigma_{1} &&  && 0 \\  && \ddots &&  \\ 0 &&  && \sigma_{P} \\ 0 && \cdots && 0 \end{bmatrix}
		 \begin{bmatrix} v_{11} && \cdots && v_{1P} \\ \vdots && \ddots && \vdots \\ v_{P1} && \cdots && v_{PP} \end{bmatrix}^T \\
	&= 	\begin{bmatrix} \sigma_{1} &&  && 0 \\  && \ddots &&  \\ 0 &&  && \sigma_{P} \\ 0 && \		\cdots && 0 \end{bmatrix}
		\begin{bmatrix} v_{11} && \cdots && v_{P1} \\ \vdots && \ddots && \vdots \\ v_{1P} && \cdots && v_{PP} \end{bmatrix} \\
	&= \begin{bmatrix}
		\sigma_1 {\bm{v}^{(1)}}^T  \\
		\sigma_2 {\bm{v}^{(2)}}^T  \\
		\vdots \\
		\sigma_P {\bm{v}^{(P)}}^T  \\
		\end{bmatrix}
	= \begin{bmatrix}
		\text{$P$ scalar multiplies} \\
		\text{$P$ scalar multiplies} \\
		\vdots \\ 
		\text{$P$ scalar multiplies}
		\end{bmatrix}
	= \text{$P^2$ operations}
	\end{align*}
On the other hand,
	\begin{align*}
		A = U^T X &= \begin{bmatrix} u_{11} && \cdots && u_{N1} \\ \vdots && \ddots && \vdots \\ u_{1P} && \cdots && u_{NP} \end{bmatrix}^T 
			 \begin{bmatrix} x_{11} && \cdots && x_{N1} \\ \vdots && \ddots && \vdots \\ x_{1P} && \cdots && x_{NP} \end{bmatrix} \\
		&= \begin{bmatrix} u_{11} && \cdots && u_{P1} \\ \vdots && \ddots && \vdots \\ u_{1N} && \cdots && u_{PN} \end{bmatrix} 
			 \begin{bmatrix} x_{11} && \cdots && x_{N1} \\ \vdots && \ddots && \vdots \\ x_{1P} && \cdots && x_{NP} \end{bmatrix} \\
		&= \begin{bmatrix}
			\text{$P$ multiplies, $P-1$ additions} && \cdots && \text{$P$ multiplies, $P-1$ additions} \\
			\vdots && \ddots && \vdots \\
			\text{$P$ multiplies, $P-1$ additions} && \cdots && \text{$P$ multiplies, $P-1$ additions}
			\end{bmatrix}_{N \times N} \\
		&= \begin{bmatrix}
			\text{$2P-1$ operations} && \cdots && \text{$2P-1$ operations} \\
			\vdots && \ddots && \vdots \\
			\text{$2P-1$ operations} && \cdots && \text{$2P-1$ operations}
			\end{bmatrix}_{N \times N} \\
		&= \text{$(2P-1)N^2$ operations}
	\end{align*}
}
\end{homeworkSection}

\end{section}


\newpage
\begin{section}{Computing}
The computing assignment is to apply the \textit{snapshot} method to a collection of high-resolution files. We will briefly discuss the background being the method, the implementation, and provide results on a test data set.
\\
\\
Suppose we have a set of $P$ $N \times N$ matrices where $P << N$. The KL expansion as discussed in class gives rise to a construction of an optimal basis for a set of vectors $\lbrace{\bm{x}^{(\mu)}} \rbrace_{\mu=1}^P$ characterized by:

\begin{align}
	C \phi^{(i)} &= \lambda_i \phi^{(i)} \label{eqn:KL}
\end{align}
\begin{align*}
	\text{where} \quad C &= \frac{1}{P} \Sigma_{\mu=1}^P (x^{(\mu)} - \langle x \rangle )(x^{(\mu)} - \langle x \rangle )^T \qquad \text{is the ensemble average covariance matrix} \\
	\text{and} \quad \langle x \rangle &= \frac{1}{P} \Sigma_{\mu=1}^P x^{(\mu)} \qquad \text{is the ensemble average}
\end{align*}

Notice that $C$ is $N \times N$. When $N$ becomes large, it is not feasible to solve this problem directly. If $C$ is nonsingular, we can reduce (without approximation) the problem from Equation \ref{eqn:KL} into a $P \times P$ problem. This is known as the $snapshot$ method.

\vspace{10pt}

\problemAnswer{
	\vspace{10pt}
	The test data set in question involves a fixed camera in a room with a person facing the camera and moving in the foreground. The ensemble average of these data is shown in Figure \ref{fig:ensemble_average}.

    \begin{minipage}{1.0\textwidth}
        \begin{figure}[H]
        \centering
        \includegraphics[trim={0cm 2cm 0cm 2cm},clip,width=0.85\columnwidth]{../data/ensemble_average}
        \caption{Ensemble Average of data set}
        \label{fig:ensemble_average}
        \end{figure}
    \end{minipage}
}
\problemAnswer{
	\vspace{10pt}
	Next, we display one of the mean-subtracted images, that is, the data set minus the ensemble average (Figure \ref{fig:mean_subtracted_ex}). \\
    \begin{minipage}{1.0\textwidth}
        \begin{figure}[H]
        \centering
        \includegraphics[trim={2cm 3cm 2cm 3cm},clip,width=0.85\columnwidth]{../data/mean_subtracted_ex}
        \setlength{\belowcaptionskip}{15pt}
        \caption{One mean-subtracted image}
        \label{fig:mean_subtracted_ex}
        \end{figure}
    \end{minipage}
    \\
    Additionally, several eigen-images are shown below:

    \begin{minipage}{1.0\textwidth}
        \begin{figure}[H]
        \centering
        \includegraphics[trim={0cm 0cm 0cm 2cm},clip,width=0.85\columnwidth]{../data/eig_images}
        \setlength{\abovecaptionskip}{-20pt}
        \caption{Some eigen-images}
        \label{fig:eig_images}
        \end{figure}
    \end{minipage}
    Next, we show the cumulative energy (as defined in a previous homework) of the mean-subtracted data set (Figure \ref{fig:cumulative_energy}). 
}

\problemAnswer{
    \begin{minipage}{1.0\textwidth}
        \begin{figure}[H]
        \centering
        \includegraphics[trim={0cm 0cm 0cm 0cm},clip,width=0.85\columnwidth]{../data/cumulative_energy}
        \caption{Cumulative Energy}
        \label{fig:cumulative_energy}
        \end{figure}
    \end{minipage}}

\end{section}

%----------------------------------------------------------------------------------------
\newpage

\appendix

\section{Code}\label{code}

\subsection{Gram-Schmidt} \label{code:gram_schmidt}

%\lstinputlisting{../Kristin_Holmbeck_HW2_GramSchmidt.m}


\begin{thebibliography}{10}
    \bibitem{chang}
    Chang, Jen-Mei. \textit{Matrix Methods for Geometric Data Analysis and Recognition}. 2014.

\end{thebibliography}

\end{document}
