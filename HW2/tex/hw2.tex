%%%%%%%%%%%%%%%%%%%%%%%%%%%%%%%%%%%%%%%%%
% Programming/Coding Assignment
% LaTeX Template
%
% This template has been downloaded from:
% http://www.latextemplates.com
%
% Original author:
% Ted Pavlic (http://www.tedpavlic.com)
%
% Note:
% The \lipsum[#] commands throughout this template generate dummy text
% to fill the template out. These commands should all be removed when 
% writing assignment content.
%
% This template uses a Perl script as an example snippet of code, most other
% languages are also usable. Configure them in the "CODE INCLUSION 
% CONFIGURATION" section.
%
%%%%%%%%%%%%%%%%%%%%%%%%%%%%%%%%%%%%%%%%%

%----------------------------------------------------------------------------------------
%	PACKAGES AND OTHER DOCUMENT CONFIGURATIONS
%----------------------------------------------------------------------------------------

\documentclass{article}

\usepackage{fancyhdr} % Required for custom headers
\usepackage{lastpage} % Required to determine the last page for the footer
\usepackage{extramarks} % Required for headers and footers
\usepackage[usenames,dvipsnames]{color} % Required for custom colors
\usepackage{graphicx} % Required to insert images
\usepackage{listings} % Required for insertion of code
\usepackage{courier} % Required for the courier font
\usepackage{lipsum} % Used for inserting dummy 'Lorem ipsum' text into the template

\usepackage{float}
\usepackage{amsfonts}
\usepackage{amsmath}

% Margins
\topmargin=-0.45in
\evensidemargin=0in
\oddsidemargin=0in
\textwidth=6.5in
\textheight=9.0in
\headsep=0.25in

\linespread{1.1} % Line spacing

% Set up the header and footer
\pagestyle{fancy}
\lhead{\hmwkAuthorName} % Top left header
\chead{\hmwkClass\ : \hmwkTitle} % Top center head
\rhead{\firstxmark} % Top right header
\lfoot{\lastxmark} % Bottom left footer
\cfoot{} % Bottom center footer
\rfoot{Page\ \thepage\ of\ \pageref{LastPage}} % Bottom right footer
\renewcommand\headrulewidth{0.4pt} % Size of the header rule
\renewcommand\footrulewidth{0.4pt} % Size of the footer rule

\setlength\parindent{0pt} % Removes all indentation from paragraphs

%----------------------------------------------------------------------------------------
%	CODE INCLUSION CONFIGURATION
%----------------------------------------------------------------------------------------

\usepackage{color} %red, green, blue, yellow, cyan, magenta, black, white
\definecolor{mygreen}{RGB}{28,172,0} % color values Red, Green, Blue
\definecolor{mylilas}{RGB}{170,55,241}

\lstset{language=Matlab,%
    basicstyle=\ttfamily\footnotesize,breaklines=true
    %basicstyle=\footnotesize\color{red},
    breaklines=true,%
    xleftmargin=0.5in,
    %xrightmargin=0.25in,
    morekeywords={matlab2tikz},
    keywordstyle=\color{blue},%
    morekeywords=[2]{1}, keywordstyle=[2]{\color{black}},
    identifierstyle=\color{black},%
    stringstyle=\color{mylilas},
    commentstyle=\color{mygreen},%
    showstringspaces=false,%without this there will be a symbol in the places where there is a space
    numbers=left,%
    numberstyle={\tiny \color{black}},% size of the numbers
    numbersep=9pt, % this defines how far the numbers are from the text
    emph=[1]{for,end,break},emphstyle=[1]\color{blue}, %some words to emphasise
    %emph=[2]{word1,word2}, emphstyle=[2]{style},    
}


%----------------------------------------------------------------------------------------
%	DOCUMENT STRUCTURE COMMANDS
%	Skip this unless you know what you're doing
%----------------------------------------------------------------------------------------

% Header and footer for when a page split occurs within a problem environment
\newcommand{\enterProblemHeader}[1]{
%\nobreak\extramarks{#1}{#1 continued on next page\ldots}\nobreak
%\nobreak\extramarks{#1 (continued)}{#1 continued on next page\ldots}\nobreak
}

% Header and footer for when a page split occurs between problem environments
\newcommand{\exitProblemHeader}[1]{
\nobreak\extramarks{#1 (continued)}{#1 continued on next page\ldots}\nobreak
\nobreak\extramarks{#1}{}\nobreak
}

\setcounter{secnumdepth}{0} % Removes default section numbers
\newcounter{homeworkProblemCounter} % Creates a counter to keep track of the number of problems

\newcommand{\homeworkProblemName}{}
\newenvironment{homeworkProblem}[1][Problem \arabic{homeworkProblemCounter}]{ % Makes a new environment called homeworkProblem which takes 1 argument (custom name) but the default is "Problem #"
\stepcounter{homeworkProblemCounter} % Increase counter for number of problems
\renewcommand{\homeworkProblemName}{#1} % Assign \homeworkProblemName the name of the problem
\subsection{\homeworkProblemName} % Make a section in the document with the custom problem count
\enterProblemHeader{\homeworkProblemName} % Header and footer within the environment
}{
\exitProblemHeader{\homeworkProblemName} % Header and footer after the environment
}

\newcommand{\problemAnswer}[1]{ % Defines the problem answer command with the content as the only argument
\noindent\framebox[\columnwidth][c]{\begin{minipage}{0.98\columnwidth}#1\end{minipage}} % Makes the box around the problem answer and puts the content inside
}

\newcommand{\homeworkSectionName}{}
\newenvironment{homeworkSection}[1]{ % New environment for sections within homework problems, takes 1 argument - the name of the section
\renewcommand{\homeworkSectionName}{#1} % Assign \homeworkSectionName to the name of the section from the environment argument
\subsection{\homeworkSectionName} % Make a subsection with the custom name of the subsection
\enterProblemHeader{\homeworkProblemName\ [\homeworkSectionName]} % Header and footer within the environment
}{
\enterProblemHeader{\homeworkProblemName} % Header and footer after the environment
}


%----------------------------------------------------------------------------------------
%	NAME AND CLASS SECTION
%----------------------------------------------------------------------------------------

\newcommand{\hmwkTitle}{Homework\ 2} % Assignment title
\newcommand{\hmwkDueDate}{Thursday,\ February\ 15,\ 2018} % Due date
\newcommand{\hmwkClass}{Math\ 521} % Course/clas
\newcommand{\hmwkAuthorName}{Kristin Holmbeck} % Your name

%----------------------------------------------------------------------------------------
%	TITLE PAGE
%----------------------------------------------------------------------------------------

\title{
\vspace{2in}
\textmd{\textbf{\hmwkClass:\ \hmwkTitle}}\\
\normalsize\vspace{0.1in}\small{Due\ on\ \hmwkDueDate}\\
\vspace{0.1in}
\vspace{3in}
}

\author{\textbf{\hmwkAuthorName}}
\date{} % Insert date here if you want it to appear below your name

%----------------------------------------------------------------------------------------

\begin{document}

\maketitle

%----------------------------------------------------------------------------------------
%	TABLE OF CONTENTS
%----------------------------------------------------------------------------------------

%\setcounter{tocdepth}{1} % Uncomment this line if you don't want subsections listed in the ToC

\newpage
\tableofcontents
\listoffigures
\newpage

%----------------------------------------------------------------------------------------
%	PROBLEM 1
%----------------------------------------------------------------------------------------

% To have just one problem per page, simply put a \clearpage after each problem

\begin{section}{Theory}

\begin{homeworkSection}{1. Unique Decomposition}

Let $W_1, W_2$ be vector subspaces and $W=W_1+W_2, W_1 \neq W_2$. Show, by giving an example, that the decomposition of a vector $\bm{x} \in W$ is not unique.

\problemAnswer{
    The requirements for a subspace $\hat{W}$ include :
    \begin{itemize}
        \item The zero vector is in $\hat{W}$
        \item If $\bm{u}, \bm{v} \in \hat{W}$, then $\bm{u}+\bm{v} \in \hat{W}$
        \item If $\bm{u} \in \hat{W}, c \in \mathbb{R}, c\bm{u} \in \hat{W}$
    \end{itemize}
    If we let $W = \{\begin{bmatrix} x && y && 0 \end{bmatrix}^T \ni x,y \in \mathbb{R} \} \subset \mathbb{R}^3$, then we can decompose $W$ into $W_1 = \{\begin{bmatrix} x && 0 && 0 \end{bmatrix}^T \ni x,y \in \mathbb{R} \}$ and $W_2 = \{\begin{bmatrix} x && y && 0 \end{bmatrix}^T \ni x,y \in \mathbb{R} \}$. Then, we can express a vector $\bm{x} \in W$ non-uniquely. For example:
    \begin{align*}
        \bm{x} = \begin{bmatrix} 2 \\ 10 \\ 0 \end{bmatrix} 
            = \begin{bmatrix} 1 \\ 0 \\ 0 \end{bmatrix} + \begin{bmatrix} 1 \\ 10 \\ 0 \end{bmatrix} 
            = \begin{bmatrix} 2 \\ 0 \\ 0 \end{bmatrix} + \begin{bmatrix} 0 \\ 10 \\ 0 \end{bmatrix}
    \end{align*}
}

\end{homeworkSection}

\begin{homeworkSection}{2. Bases for $A$}
Consider the matrix
\begin{align*}
    A = \begin{bmatrix} 1 && -1 \\ 2 && -2 \\ 3 && -3 \end{bmatrix}
\end{align*}
Determine bases for the column space, row space, null space, and left null space of $A$.

\rule{\textwidth}{1pt}

\problemAnswer{ 

\begin{itemize}

\item The column space of $A$ is the linearly independent columns in $A$. Since the second column is a scalar multiple of the first (by -1), the column space of $A$ is: \\ span$ \left \{ \begin{bmatrix} 1 \\2 \\ 3 \end{bmatrix} \right \} $.

\item The row space of $A$ is the linearly independent rows of $A$. Notice that rows 2 and 3 are scalar multiples of the first row. Hence, the row space is: \\ span$ \left \{ \begin{bmatrix} 1 \\ -1 \end{bmatrix} \right \} $.

\item The null space of $A$ includes the vectors that solve $Ax = \bm{0}$. Then it is easy to see that $A \begin{bmatrix} x_1 \\ x_1 \end{bmatrix} = \bm{0}$ solves this. In other words, \\ null space of $A$ = span $\left\{ \begin{bmatrix} 1 \\ 1 \end{bmatrix} \right \}$.

\item The \textit{left} null space of $A$ are the vectors that solve $x^TA = 0$. \\
$\begin{bmatrix} x_1 & x_2 && x_3 \end{bmatrix} \begin{bmatrix} 1 && -1 \\ 2 && -2 \\ 3 && -3 \end{bmatrix} = \begin{bmatrix} x_1+2x_2+3x_3 && -x_1-2x_2-3x_3 \end{bmatrix} = \bm{0}$ \\
The most direct way to solve this is to convert $ [ A \quad | \quad I_{3}  ] $ to reduced-row echelon form. Performing this calculation, we obtain the basis for the left null space: \\
span = $\left \{ \begin{bmatrix} 1 \\ 0 \\ -\frac{1}{3}\end{bmatrix},
                \begin{bmatrix} 0 \\ 1 \\ -\frac{2}{3}\end{bmatrix} \right \}
        $
\end{itemize}
}

\end{homeworkSection}


\begin{homeworkSection}{3. Projections}
Let $V = \mathbb{R}^3$, let 
\begin{align*}
    u^{(1)} = \begin{bmatrix} 1 \\ 2 \\ 0 \end{bmatrix}, 
    u^{(2)} = \begin{bmatrix} -1 \\ 0 \\ 1 \end{bmatrix}, 
    x = \begin{bmatrix} 0 \\ 2 \\ 1 \end{bmatrix},
\end{align*}
and define $W = \text{span}(u^{(1)}, u^{(2)})$. Find the orthogonal projection of $x$ onto $W$. Also find the projection matrix $\mathbb{P}$ associated with this mapping.

\problemAnswer{
    For orthogonal projections, the Gram-Schmidt process is used. The Gram-Schmidt process is a method for generating an orthonormal basis from a set of vectors. A key part of this method is the projection step.
    \\
    First, we determine an orthonormal basis for $W$ using the Gram-Schmidt algorithm (see the code in the Code section). We obtain the basis vectors:
    \begin{align*}
        e^{(1)} = \frac{1}{\sqrt{5}} \begin{bmatrix} 1 \\ 2 \\ 0 \end{bmatrix}, \qquad
        e^{(2)} = \frac{1}{3\sqrt{5}} \begin{bmatrix} -4 \\ 2 \\ 5 \end{bmatrix}
    \end{align*}
    A projection matrix $\mathbb{P}_i$ onto $e^{(i)}$, is given by $e^{(i)} {e^{(i)}}^T$. From this, we have the two projection matrices for the basis vectors above:
    \begin{align*}
        \mathbb{P}_1 &= \frac{1}{5} \begin{bmatrix} 1 && 2 && 0 \\ 2&& 4 && 0 \\ 0 && 0 && 0 \end{bmatrix} \\ 
        \mathbb{P}_2 &= \frac{1}{45} \begin{bmatrix} 16 && -8 && -20 \\ -8 && 4 && 10 \\ -20 && 10 && 25 \end{bmatrix} \\ 
        \text{then the total projection is: }
        \mathbb{P} &= \mathbb{P}_1 + \mathbb{P}_2 = \frac{1}{9} \begin{bmatrix} 5 && 2 && -4 \\ 2 && 8 && 2 \\ -4 && 2 && 5 \end{bmatrix}
    \end{align*}
}
\end{homeworkSection}

\begin{homeworkSection}{4. Orthonormal Basis Vectors}
Reconsider Problem 3. Find vectors such that $x = UU^Tx$ and $x \neq UU^Tx$ where the matrix $U$ consists of the orthonormal basis vectors of $W$ from Problem 3.

\problemAnswer{
    Note that $UU^T = \mathbb{P}$.
}
\end{homeworkSection}

\begin{homeworkSection}{5. SVD}
Determine the SVD of the data matrix
\begin{align*}
    A = \begin{bmatrix} -2 && -1 && 1 \\ 0 && -1 && 0 \\ -1 && 1 && 2 \\ 1 && -1 && 1 \end{bmatrix}
\end{align*}
and compute rank-one, -two, and -three approximations to $A$.
\end{homeworkSection}

\end{section}

%----------------------------------------------------------------------------------------
%	PROBLEM 2
%----------------------------------------------------------------------------------------
\begin{section}{Computing}

\begin{homeworkSection}{1. Kohonen's Novelty Filter }
Consider the training set consisting of the following three patterns consisting of 5 x 4 arrays of black squares.
\\

Proceed by assuming that the black square entries have numerical value one and the blank entries have numerical value zero. Concatenate the columns of each pattern to make vectors in $\mathbb{R}^{20}$. Does your result make sense? Why or why not?

% \problemAnswer{} 

\end{homeworkSection}


\begin{homeworkSection}{2. SVD}
Compute the SVD of the matrix $A$ whose entries come from the pattern in <reference here> and display (in terms of an image) the reconstructions $A_1, A_2, A_3, A_4$. Again, treat the squares as ones and the blanks as zeros. Your reconstructions should be matrices with numerical values. Interpret your results.

\end{homeworkSection}

\end{section}


%----------------------------------------------------------------------------------------
\newpage

\appendix

\section{Code}

\subsection{Gram-Schmidt} \label{code:gram_schmidt}

\lstinputlisting{../Kristin_Holmbeck_HW2_GramSchmidt.m}


\begin{thebibliography}{10}
    \bibitem{kohonen}
    T. Kohonen. Self-Organization and Associative Memory. Springer-Verlag, Berlin, 1984.

\end{thebibliography}

\end{document}